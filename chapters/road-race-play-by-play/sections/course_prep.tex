\section{Course Preparation}

In the months, weeks, and days leading up to the race, some on-site work may need to be done.

According to USA Cycling regulations, the \pbproleref{role:chief_ref} may request to preview the course well before the race.
This allows the \pbproleref{role:chief_ref} to assess the safety and design of the event.

Regardless of the official's preview, the \pbproleref{role:lead_org} should ensure that someone previews the course 5-7 days before the event.
This preview should check:
\begin{itemize}
  \item The state of potholes, sewer grates, and other hazards that may need to be mitigated
  \item The presence of any construction, detours, or other obstacles that may require changes to the course
  \item A final review of where marshals may be needed, and where the finish line should be located
\end{itemize}

\subsubsection{Signage and Notifying Residents}

The \pbproleref{role:lead_org} should coordinate with the local police and find out how local residents should be notified of the race.

Often the \pbproleref{role:race_org_team} will need to drop off a letter at each residence (working with \pbproleref{role:local_teams} if needed),
but sometimes the police will suggest using social media or reverse 911 to contact residents.

Additionally, the \pbproleref{role:lead_org} should check if no-parking signs can be put up.
Often the no-parking signs should be put up the night before the race, or one day in advance.

% TODO: any legal notes?

The \pbproleref{role:lead_org} should consider other local signage, such as advertising signs that could be put up in the town,
or signs directing arriving participants to the venue.

\subsubsection{Barriers}

Well before the race, the \pbproleref{role:lead_org} should coordinate with the local police and/or highway department
to ensure that barriers can be placed around the race course.
For a criterium, often the entire course is shut down with barriers, police officers, and marshals.
With a road race course, barriers are less often used (as the entire course is typically open to traffic), but barriers can be used
as ``checkpoints'' to control the traffic flow as cars approach the course.

In some cases, the police or highway department will require that they setup the barriers themselves (often at an additional cost),
as road closures are an official matter.

Other times, the police will ask the \pbproleref{role:lead_org} to move barriers the night before the race,
so the detail officers just need to move the barriers into position when they arrive on the day of the event.
